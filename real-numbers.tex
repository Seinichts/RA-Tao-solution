\begin{exercise}[ (Cauchy sequences are bounded)]
  Every Cauchy sequence $(a_n)_{n=1}^\infty$ is bounded.
\end{exercise}
\begin{proof}
  We can split the sequence into two parts, $a_1, \dots, a_{N-1}$ and $a_N, a_{N+1}, \dots$ and prove that both split are bounded.
  Since the first split is a finite sequence, it is bounded. 

  For the second split, by the definition of Cauchy sequence, $d(a_j, a_k) < \varepsilon$ for any $j, k \geq N$ and for any $\varepsilon > 0$.
  Let $M = a_N + 1$ and $m = a_N - 1$, then $d(a_j, M) < \varepsilon + 1$ and $d(a_j, m) < \varepsilon + 1$,
  which implies that $\forall j \geq N, M > a_j$ and $\forall j \geq N, M > a_j$. We end the proof by letting $M_{\text{bound} = \max{\abs{M}, \abs{m}}}$.
  One of them will give the desired result.
\end{proof}


\begin{exercise}
  Show that if $(a_n)_{n=1}^\infty$ and $(b_n)_{n=1}^\infty$ are equivalent sequences of rationals,
  then $(a_n)_{n=1}^\infty$ is a Cauchy sequence if and only if $(b_n)_{n=1}^\infty$ is a Cauchy sequence.
\end{exercise}
\begin{proof}
  Suppose $(a_n)_{n=1}^\infty$ and $(b_n)_{n=1}^\infty$ are equivalent sequences of rationals,
  then there exists a rational number $r$ such that $d(a_n, b_n) \leq r$ for all $n > N$.
  If $(a_n)_{n=1}^\infty$ is a cauchy sequence, then we have $d(a_j, a_k) < \varepsilon$ for any $j, k > N$ and for any $\varepsilon > 0$.
  Then for any $j, k \geq N$, $$r \geq r_j - \varepsilon + \varepsilon - r_k = d(b_j, a_j) - d(a_j, a_k) + d(a_j, a_k)- d(a_k, b_k) = d(b_j, b_k).$$
  This implies that $(b_n)_{n=1}^\infty$ is a Cauchy sequence. The other direction is similar.
\end{proof}



\begin{exercise}
  Let $\varepsilon > 0$. Show that if $(a_n)_{n=1}^\infty$ and $(b_n)_{n=1}^\infty$ are eventually $\varepsilon$-close,
  then $(a_n)_{n=1}^\infty$ is bounded if and only if $(b_n)_{n=1}^\infty$ is bounded.
\end{exercise}
\begin{proof}
  Suppose $(a_n)_{n=1}^\infty$ is bounded, then it is equivalent to $\forall a_n, \abs{a_n} \leq M$. 
  Then we only have to show the part of $(a_n)$ where $n \geq N$ are bounded implies the part of $(b_n)$ where $n \geq N$ are also bounded.
  For both $(a_n) \text{\ and\ } (b_n)$, the part where $n \leq N-1$ are bounded.
  Since $(a_n)_{n=1}^\infty$ and $(b_n)_{n=1}^\infty$ are eventually $\varepsilon$-close, that is, $\exists N \geq 0, \forall n \geq N, d(a_n, b_n) < \varepsilon$.
  We can take $M = \max\{M_a, M_b\}$, where $M_a$ is the bound of $(a_n)$ and $M_b$ is the bound of $(b_n)$.
  Then $\forall n \geq N, \abs{b_n} \leq M + \varepsilon$. Therefore $(b_n)_{n=1}^\infty$ is bounded. The other direction is similar.
\end{proof}



\begin{exercise}
  Prove that Formal limits are well-defined: Let $x = \lim_{n\to \infty}a_n, y = \lim_{n\to \infty} \text{\ and\ } z = \lim_{n \to \infty}c_n$
  be real numbers. Then, with the above definition of equality for real numbers, we have $x = x$. Also, if $x = y$ and $y = z$, then $x = z$.
\end{exercise}

\begin{proof}
  Obviously, $(a_n)$ is 0-close to it self, hence we could say that $(a_n)$ is equivalent to $(a_n)$ and thus $\lim (a_n) = x = x$. \par
  Suppose $x = y$, then we have $\lim (a_n) = \lim (b_n)$, which means that $(a_n)$ is equivalent to $(b_n)$. 
  That is, $(a_n)$ will eventually be $\varepsilon$-close to $(b_n)$. Therefore, $b = a$ \par
  Now, if $x = y$ and $y = z$, then we have $d(a_n, b_n)$ and $d(b_n, c_n)$ are $\varepsilon$-close and will eventually be $\frac{\varepsilon}{2}$-close
  Therefore, $d(a_n, c_n) \leq \frac{\varepsilon}{2} + \frac{\varepsilon}{2} = \varepsilon$, we conclude that if $x = y$ and $y = z$, then $x = z$.
\end{proof}



\begin{exercise}
  Prove that multiplication is well-defined: Let $x = \lim_{n \to \infty}a_n$, $y = \lim_{n \to \infty} b_n$, and $x' = \lim_{n \to \infty} a'_n$ be real numbers.
  Then $xy$ is also a real number. Furthermore, if $x = x'$, then $xy = x'y$.
\end{exercise}
\begin{proof}
  Since $x \text{\ and\ } y$ are real numbers and such that $(a_n) \text{\ and\ } (b_n)$ are Cauchy sequences, there exists $M_1 \geq \abs{(a_n)}$ and $M_2 \geq \abs{(b_n)}$.
Let $\varepsilon \definedas \min (\frac{\varepsilon'}{3M_2}, \frac{\varepsilon'}{3})$ for which $(a_n)$ is eventually $\varepsilon$-steady. And let
$\delta \definedas \min (\frac{\varepsilon'}{3M_1}, 1)$ for which $(b_n)$ is eventually $\delta$-steady.
We know that $a_nb_n$ are eventually $(\varepsilon \abs{b_n} + \delta \abs{a_n} + \varepsilon \delta)$-close. And also
\begin{align*}
\varepsilon \abs{b_n} \leq \varepsilon M_2 \leq \frac{\varepsilon'}{3M_2} \cdot M_2 &= \frac{\varepsilon'}{3} \\
\delta \abs{a_n} \leq \delta M_1 \leq \frac{\varepsilon'}{3M_1} \cdot M_1 &= \frac{\varepsilon'}{3} \\
\varepsilon \delta \leq \frac{\varepsilon'}{3} \cdot 1 = \frac{\varepsilon'}{3}
.\end{align*}
So, $a_nb_n$ are eventually $\varepsilon'$-close. Therefore, $xy$ is a real number. \par
\end{proof}

